\chapter{Installation und Einrichtung}

Das Kommandozeilen-Tool \pica{} lässt sich unter den Betriebssystemen
Linux, macOS und Windows nutzen. Folgend wird die Installation von
sowie Einrichtung und Konfiguration des Tools beschrieben.

\section{Installation}

\subsection{Installation unter Linux}

Abhängig von der genutzten Linux-Distribution, gibt es
unterschiedliche Möglichkeiten Möglichkeiten der
Installation. Vorgefertigte Releases stehen auf der Plattform GitHub
zum Download
bereit\footnote{\url{https://github.com/deutsche-nationalbibliothek/pica-rs/releases}}.

\subsubsection{Debian und Ubuntu}

Unter \href{https://www.debian.org}{Debian GNU/Linux} und
\href{https://ubuntu.com}{Ubuntu Linux} können fertige
\texttt{DEB}-Pakete genutzt werden.  Diese lassen sich mit dem
\lit*{dpkg}-Programm wie folgt installiert werden:

\begin{minted}{shell-session}
$ dpkg -i pica_0.25.0-glibc2.35-1_amd64.deb
\end{minted}

\subsubsection{Red Hat, SUSE und CentOS}

Für die Distributionen \href{https://www.redhat.com}{Red Hat Linux},
\href{https://www.suse.com}{SUSE Linux} und
\href{https://www.centos.org}{CentOS Linux} stehen fertige
\texttt{RPM}-Pakete zum Download bereit. Installieren lassen sich
diese mit dem \texttt{rpm}-Programm installieren:

\begin{minted}{shell-session}
$ rpm -i pica-0.25.0-glibc2.35-1.x86_64.rpm
\end{minted}

\subsubsection{Binary Releases}

Soll \pica{} nicht über den Paketmanager installiert werden, steht für
die Zielarchitektur \texttt{x86_64-unknown-linux-gnu} mit den
\texttt{glibc}-Versionen 2.28, 2.31 und 2.35 fertige \emph{Binary
  Releases} zur Verfügung\footnote{Die \texttt{glibc}-Version des
  Systems lässt sich mit dem Aufruf \texttt{ldd ----version}
  ermitteln.}.

Das \texttt{tar}-Archiv enthält neben dem Tool \pica{} auch weitere
Dateien wie die Shell-Skripte zur Befehlszeilenergänzung:

\begin{minted}{shell-session}
$ tar -tf pica-0.25.0-x86_64-unknown-linux-gnu-glibc2.35.tar.gz
pica-0.25.0-x86_64-unknown-linux-gnu-glibc2.35/
pica-0.25.0-x86_64-unknown-linux-gnu-glibc2.35/pica
pica-0.25.0-x86_64-unknown-linux-gnu-glibc2.35/README.md
pica-0.25.0-x86_64-unknown-linux-gnu-glibc2.35/pica.zsh
pica-0.25.0-x86_64-unknown-linux-gnu-glibc2.35/LICENSE
pica-0.25.0-x86_64-unknown-linux-gnu-glibc2.35/pica.fish
pica-0.25.0-x86_64-unknown-linux-gnu-glibc2.35/pica.bash
\end{minted}

Eine systemweite Installation von \pica{} in das Verzeichnis
\texttt{/usr/local/bin} kann mit dem \lit*{install} erfolgten. Hierfür
sind ggf. \lit*{root}-Rechte nötig:

\begin{minted}{shell-session}
$ tar xfz pica-0.25.0-x86_64-unknown-linux-gnu-glibc2.35.tar.gz
$ sudo install -m755 pica-0.25.0-x86_64-unknown-linux-gnu-glibc2.35/pica \
    /usr/local/bin/pica
\end{minted}

\subsection{Installation unter macOS}

Unter \textit{macOS} wird nur die Zielarchitektur
\texttt{x86_64-apple-darwin} (macOS 10.7+, Lion+) unterstützt. Diese
lässt sich analog wie die Linux-Release installieren:

\begin{minted}{shell-session}
$ tar xfz pica-0.25.0-x86_64-apple-darwin.tar.gz
$ install -m755  pica-0.25.0-x86_64-apple-darwin/pica /usr/local/bin/pica
\end{minted}


\subsection{Installation unter Windows}

Unter Windows (\texttt{x86_64-pc-windows-gnu} oder
\texttt{x86_64-pc-windows-msvc}) kann das Programm direkt dem
\texttt{zip}-Archiv entnommen werden. Nach einem Wechsel in das
Verzeichnis, in dem sich die \lit*{pica.exe} befindet, kann das
Programm direkt genutzt werden. Soll \pica{} aus jedem beliebigen
Verzeichnis aus aufrufbar sein, dann muss der Installationspfad in der
\lit*{PATH}-Umgebungsvariable mit aufgelistet sein.


\subsection{Aus dem Quellcode installieren}

Das Projekt lässt sich auch direkt aus den Quellen
kompilieren. Hierfür wird eine aktuelle Rust-Version ($>= 1.74.1$) mit
dem Paketmanager \lit*{cargo} benötigt.

Der aktuelle Entwicklungsstand lässt sich wie folgt installieren:

\begin{minted}{shell-session}
$ git clone https://github.com/deutsche-nationalbibliothek/pica-rs.git
$ cd pica-rs
$ cargo build --release
\end{minted}

Das fertige Release-Binary \pica{} liegt im Verzeichnis
\Verb{target/release/} und kann bspw. in das Verzeichnis
\Verb{/usr/local/bin} installiert werden:

\begin{minted}{shell-session}
$ install -m755 target/release/pica /usr/local/bin/pica
\end{minted}

Wenn die Quelle nicht benötigt werden, kann das Projekt auch direkt
über den Paketmanager \cmd{cargo} installiert werden:

\begin{minted}{shell-session}
$ # Installation der aktuellen Entwicklungsversion
$ cargo install --git https://github.com/deutsche-nationalbibliothek/pica-rs \
     --branch main pica-toolkit

$ # Installation der Version 0.25.0
$ cargo install --git https://github.com/deutsche-nationalbibliothek/pica-rs \
      --tag v0.25.0 pica-toolkit
\end{minted}

Das fertige Programm befindet sich dann unter
\Verb{~/.cargo/bin/pica}.


\section{Einrichtung}

\subsection{Shell}

Nach der Installation kann \pica{} direkt in einem
Kommandozeileninterpreter (\emph{Shell}) verwendet werden. Die
Interpretation der Kommandoaufrufe und -optionen ist abhängig von der
verwendeten Shell. Alle Beispiele in diesem Manual werden in der
\href{https://de.wikipedia.org/wiki/Bash_(Shell)}{bash}-Shell unter
Linux ausgeführt. Wie die Argumente und Kommandoaufrufe von einer
anderen Shell und dem Betriebssystemen interpretiert werden, kann
variieren und ist ggf. bei einer Fehlersuche mit einzubeziehen.

\warningbox{Insbesondere unter Windows muss in der
  \href{https://de.wikipedia.org/wiki/PowerShell}{PowerShell}
  besondere Vorsicht walten. Hier wird die Zeichenkodierung der Daten
  bei der Vekettung von Kommandos verändert. Aus diesem Grund wird von
  der Verwendung der PowerShell abgeraten und stattdessen die
  Eingabeaufforderung \lit*{cmd.exe} empfohlen. Siehe auch den GitHub
  \href{https://github.com/deutsche-nationalbibliothek/pica-rs/issues/371}{Issue
    \#371}.}

Eine wichtige Eigenschaft der verwendeten Shell, die Benutzende immer
im Hinterkopf haben sollten, ist die Interpretionen von
Zeichenketten. Hier werden ggf. besondere Anweisungen durch die Shell
ausgeführt und durch andere Ausdrücke ersetzt. Dieses Verhalten führt
zu fehlerhaften \pica{}-Anweisungen.

\tipbox{Um in der \lit*{bash}-Shell die Fehler zu minimieren, sollten
  Literale (bspw. Filter- und Selektionsausdrücke) immer in einfachen
  Anführungszeichen ausgezeichnet werden. Ebenfalls sinnvoll kann die
  Deaktivierung der History-Substitution mittels \Verb{set +H}
  sein. Dies verhindert das Ersetzen von \lit{!0}-Fragmenten durch den
  letzten Befehl, der von der Shell ausgeführt wurde.}

\subsubsection*{Befehlszeilenergänzung}

TODO

\subsection{Konfiguration}

TODO

%%% Local Variables:
%%% mode: LaTeX
%%% TeX-master: "manual"
%%% End:
