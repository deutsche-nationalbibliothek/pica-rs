\chapter{Einführung}
\label{ch:01}

Das Projekt \lit*{pica-rs} ermöglicht eine effiziente Verarbeitung von
bibliografischen Metadaten, die in PICA+, dem internen Format des
\href{https://www.oclc.org/}{OCLC}-Katalogsystems, kodiert sind. Das
Kommandozeilenprogramm \pica{} stellt unterschiedliche Kommandos zur
Verfügung, um Daten auszuwählen, statistisch aufzubereiten oder für
die Weiterverarbeitung in \emph{Data Science}-Frameworks wie
\href{https://pola.rs/}{Polars} (Python) oder der Programmiersprache
\href{https://www.r-project.org/}{R} nutzbar zu machen. Die Anwendung
ist in der Programmiersprache \textit{Rust} geschrieben und lässt sich
unter den Betriebsystemen Linux, macOS und Windows verwenden. Die
Kommandos lassen sich über die Standard-Datenströme miteinander
verketten, wodurch sich leicht Metadaten-Workflows erstellen und
automatisieren lassen.

Die Entwicklung von \lit*{pica-rs} wurde vom Referat \ac{AEN} der
\ac{DNB} initiert und wird dort für die Erstellung von Datenanalysen
sowie für die Automatisierung von Workflows (Datenmanagement) im
Rahmen der automatischen Inhaltserschließung genutzt. Weiterhin wird
es zur Unterstützung der Forschungsarbeiten im
\href{https://www.dnb.de/ki-projekt}{KI-Projekt} sowie für diverse
andere Datenanalysen innerhalb der \ac{DNB} eingesetzt.

%%% Local Variables:
%%% mode: LaTeX
%%% TeX-master: "manual"
%%% End:
